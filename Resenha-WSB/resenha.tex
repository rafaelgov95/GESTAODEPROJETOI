\documentclass[12pt]{article}
\usepackage[utf8]{inputenc}
\usepackage[T1]{fontenc}
\usepackage[brazilian]{babel}
\usepackage{verbatim}	
\usepackage{tikz}
\usepackage{smartdiagram}
\usetikzlibrary{arrows,shapes,positioning,shadows,trees}
\tikzset{
	basic/.style  = {draw, text width=4cm, drop shadow, font=\sffamily, rectangle},
	root/.style   = {basic, rounded corners=2pt, thin, align=center,
		fill=green!30},
	level 2/.style = {basic, rounded corners=6pt, thin,align=center, fill=green!60, text width=8em},
	level 3/.style = {basic, thin, align=left, fill=pink!60, text width=6.5em,minimum height=1em}
}

\title{Resenha - Work Breakdown Structure}
\author{Rafael Gonçalves de  Oliveira Viana}

\date{2º semestre de 2017}

\begin{document}

\maketitle
 WBS - "\textit{Work Breakdown Structure}",também conhecida como EAP - "Estrutura Analítica de Projetos",  trata-se de uma técnica de representação gráfica que representa o projeto como um todo, assim EAP não é apenas uma ferramenta, ela tem o papel de gerenciamento do projeto na qual se aplica a subdivisão das entregas e do trabalho.
 
Segundo definições do Guia PMBOK 5ªEdição, página 125;

 \begin{quote}
“WBS é o processo de subdivisão das entregas e do trabalho do projeto em componentes menores e de gerenciamento mais fácil."
\end{quote}

Ou sejá o processo  do desenvolvimento/evolução de um projeto deve ser tratado como um conjunto de subprojetos menores tornando assim esses mesmo mais fáceis de ser gerenciados, já que possuem escopos menores para ser alcançados.
O trabalho planejado é contido dentro dos componentes de nível mais baixo da EAP, que são chamados de pacotes de trabalho.Um pacote de trabalho pode ser usado para agrupar as atividades onde o trabalho é agendado, tem seu custo estimado, monitorado e controlado. Para alcançar seus objetivos os dados são separados em Entradas, Ferramentas e Técnicas, Sáidas.

\smartdiagramset{
	module x sep=4.8,
	back arrow distance=1.75,
	text width=3.8cm,
	back arrow disabled=true,
	additions={
		additional item width=4cm,
		additional item offset=0.85cm,
		additional item border color=red,
		additional connections disabled=false,
		additional arrow color=red,
		additional arrow tip=stealth,
		additional arrow line width=3pt,
		additional arrow style=]-latex’,
	}
}


\begin{center}
	\smartdiagram[flow diagram:horizontal]{Entrada,Ferramenta e Técnicas, Saída}
\end{center}
\textbf{As entradas são divididas em;}

	\begin{itemize}

		\item Plano de gerenciamento do escopo.
		
		\item
		Especificação do escopo do projeto.
		
		\item
		Documentação dos requisitos.
		
		\item
		Fatores ambientais da empresa.
		
		\item
		Ativos de processos organizacionais.

	\end{itemize}

\textbf{As ferramentas e técnicas são divididas em;}
	\begin{itemize}

		\item
		Decomposição e Opinião especializada.
		
		\end{itemize}


\textbf{As saídas são divididas em;}
	\begin{itemize}

		\item
		Linha de base do escopo.
		
		\item
Atualizações nos documentos do projeto.

	\end{itemize}

Sua EAP pode ter tantos níveis quanto você quiser. Por uma questão visual e de impressão é razoável que cada nível não seja decomposto em mais do que 9 itens e cada página representade até 3 níveis no seu plano de projetos, mais do que isto fica difícil a visualização.

Como exemplo será demostrado a construção de uma EAP de um equipamento IoT - \textit{Internet of Things} ou como é conhecida "Internet das Coisas", onde este possua sensores conectados, enviando informações sobre condições ambientais como estado do solo do ar e de iluminação, esses dados são encaminhados para um servidor onde são feito o tratamento/conversão dos dados em informações, que posteriormente serão recuperados por uma aplicação, está EAP resumida séria da seguinte maneira;

\vspace{0.2cm}
\begin{tikzpicture}[
level 1/.style={sibling distance=40mm},
edge from parent/.style={->,draw},
>=latex]

% root of the the initial tree, level 1
\node[root] {Caixinha do Tempo}
% The first level, as children of the initial tree
child {node[level 2] (c1) {Hardware}}
child {node[level 2] (c2) {Front-End}}
child {node[level 2] (c3) {Back-End}};

% The second level, relatively positioned nodes
\begin{scope}[every node/.style={level 3}]
\node [below of = c1, xshift=15pt] (c11) {Sensores};
\node [below of = c11] (c12) {Armazenamento};
\node [below of = c12] (c13) {Dados};

\node [below of = c2, xshift=15pt] (c21) {Gráficos};
\node [below of = c21] (c22) {Relatórios};
\node [below of = c22] (c23) {Diagnóstico};
\node [below of = c23] (c24) {Propagandas};

\node [below of = c3, xshift=15	pt] (c31) {Autenticação};
\node [below of = c31] (c32) {ACLs};
\node [below of = c32] (c33) {Rotas};
\node [below of = c33] (c34) {RNA};
\node [below of = c34] (c35) {Dados};
\end{scope}

% lines from each level 1 node to every one of its "children"
\foreach \value in {1,2,3}
\draw[->] (c1.195) |- (c1\value.west);

\foreach \value in {1,...,4}
\draw[->] (c2.195) |- (c2\value.west);

\foreach \value in {1,...,5}
\draw[->] (c3.195) |- (c3\value.west);
\end{tikzpicture}

\end{document}
