\documentclass[12pt]{article}
\usepackage[utf8]{inputenc}
\usepackage[T1]{fontenc}
\usepackage[brazilian]{babel}
\usepackage{verbatim}	
\usepackage{tikz}
\usetikzlibrary{arrows,shapes,positioning,shadows,trees}
\tikzset{
	basic/.style  = {draw, text width=2cm, drop shadow, font=\sffamily, rectangle},
	root/.style   = {basic, rounded corners=2pt, thin, align=center,
		fill=green!30},
	level 2/.style = {basic, rounded corners=6pt, thin,align=center, fill=green!60,
		text width=8em},
	level 3/.style = {basic, thin, align=left, fill=pink!60, text width=6.5em}
}

\title{Resenha - Work Breakdown Structure}
\author{Rafael Gonçalves de  Oliveira Viana}

\date{2º semestre de 2017}

\begin{document}

\maketitle
EAP também conhecida como WBS - \textit{Work Breakdown}, é uma técnica de gerenciamento de projeto na qual se aplica a subdivisão das entregas e do trabalho do projeto em itens menores e mais gerenciáveis.
Esta decomposição e sua respectiva representação ajudam os responsáveis pelo projeto a entenderem tudo o que deve ser realizado.
 
Segundo definições do Guia PMBOK 5ªEdição, página 125;

 \begin{quote}
“WBS é o processo de subdivisão das entregas e do trabalho do projeto em componentes menores e de gerenciamento mais fácil."' 
\end{quote}

Ou sejá o processo  do desenvolvimento/evolução de um projeto deve ser tratado como um conjunto de subprojetos menores tornando assim esses mesmo mais fáceis de ser gerenciados, já que possuem escopos menores para ser alcançados.
Para alcançar seus objetivos os dados são separados em Entradas, Ferramentas e Técnicas, Sáidas.
O trabalho planejado é contido dentro dos componentes de nível mais baixo da EAP, que são chamados
de pacotes de trabalho.Um pacote de trabalho pode ser usado para agrupar as atividades onde o trabalho
é agendado, tem seu custo estimado, monitorado e controlado.

\textbf{As entradas são dividas em;}
	\begin{itemize}
		\begin{itemize}
		\item
		Plano de gerenciamento do escopo
		
		\item
		Especificação do escopo do projeto
		
		\item
		Documentação dos requisitos
		
		\item
		Fatores ambientais da empresa
		
		\item
		Ativos de processos organizacionais
	\end{itemize}
	\end{itemize}

\textbf{As ferramentas e técnicas}
	\begin{itemize}
	\begin{itemize}
		\item
		Decomposição
		
		\item
		Opinião especializada
		
		\end{itemize}
	\end{itemize}

\textbf{As saídas;}
	\begin{itemize}
			\begin{itemize}
		\item
		Linha de base do escopo
		
		\item
Atualizações nos documentos do projeto
		\end{itemize}
	\end{itemize}

Sua EAP pode ter tantos níveis quanto você quiser. Por uma questão visual e de impressão é razoável que cada nível não seja decomposto em mais do que 9 itens e cada página representade até 3 níveis no seu plano de projetos. Mais do que isto fica difícil a visualição quando em folha.

\begin{tikzpicture}[
level 1/.style={sibling distance=40mm},
edge from parent/.style={->,draw},
>=latex]

% root of the the initial tree, level 1
\node[root] {Drawing diagrams}
% The first level, as children of the initial tree
child {node[level 2] (c1) {Defining node and arrow styles}}
child {node[level 2] (c2) {Positioning the nodes}}
child {node[level 2] (c3) {Drawing arrows between nodes}};

% The second level, relatively positioned nodes
\begin{scope}[every node/.style={level 3}]
\node [below of = c1, xshift=15pt] (c11) {Setting shape};
\node [below of = c11] (c12) {Choosing color};
\node [below of = c12] (c13) {Adding shading};

\node [below of = c2, xshift=15pt] (c21) {Using a Matrix};
\node [below of = c21] (c22) {Relatively};
\node [below of = c22] (c23) {Absolutely};
\node [below of = c23] (c24) {Using overlays};

\node [below of = c3, xshift=15pt] (c31) {Default arrows};
\node [below of = c31] (c32) {Arrow library};
\node [below of = c32] (c33) {Resizing tips};
\node [below of = c33] (c34) {Shortening};
\node [below of = c34] (c35) {Bending};
\end{scope}

% lines from each level 1 node to every one of its "children"
\foreach \value in {1,2,3}
\draw[->] (c1.195) |- (c1\value.west);

\foreach \value in {1,...,4}
\draw[->] (c2.195) |- (c2\value.west);

\foreach \value in {1,...,5}
\draw[->] (c3.195) |- (c3\value.west);
\end{tikzpicture}

\end{document}
