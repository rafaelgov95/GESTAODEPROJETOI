\documentclass[12pt]{article}
\usepackage[utf8]{inputenc}
\usepackage[T1]{fontenc}
\usepackage[brazilian]{babel}
\usepackage{verbatim}

\title{Work Breakdown Structure - Resenha}
\author{Rafael Gonçalves de  Oliveira Viana}

\date{2º semestre de 2017}

\begin{document}

\maketitle
Segundo definições do Guia PMBOK;

 \begin{quote}
“WBS é o processo de subdivisão das entregas e do trabalho do projeto em componentes menores e de gerenciamento mais fácil. 
\end{quote}

Ou sejá o processo  do desenvolvimento/evolução de um projeto deve ser tratado como um conjunto de subprojetos menores tornando assim esses mesmo mais faceis de ser gerenciados, já que possuem escopo menores para ser alcançados





 A WBS é uma decomposição hierárquica orientada às entregas do trabalho a ser executado pela equipe para atingir os objetivos do projeto e criar as entregas requisitadas, sendo que cada nível descendente da WBS representa uma definição gradualmente mais detalhada da definição do trabalho do projeto. A WBS organiza e define o escopo total e representa o trabalho especificado na atual declaração do Escopo do projeto aprovado. O trabalho planejado é contido dentro dos componentes de nível mais baixo da WBS, que são chamados de pacotes de trabalho. Um pacote de trabalho pode ser agendado, ter seu custo estimado, monitorado e controlado. No contexto da WBS, o trabalho se refere a produtos de trabalho ou entregas que são resultado do esforço e não o próprio esforço. “

A WBS é a fase mais importante para o Gerente de Projetos. A sua elaboração requer um uma decomposição do trabalho do projeto em partes menores. A sua decomposição é chamada de estruturada “TOP-DOWN” orientada as entregas chamada de “deliverables”.

A WBS serve como base e orientação no planejamento de projeto.

Após a criação da WBS no projeto, o planejamento do projeto está todo orientado e estruturado para a equipe de execução do projeto, bem como para as demais partes interessadas tais como clientes e fornecedores.

Para isso a sua elaboração requer do Gerente de Projeto e Equipe o conhecimento de suas Entradas, Saídas e Ferramentas \& Técnicas.

Segundo o Guia PMBOK®:

Entradas – Declaração do escopo do Projeto, Documento dos Requisitos, Ativos de Processos organizacionais.

Ferramentas \& Técnicas – Decomposição.

Saídas – WBS, Dicionário da WBS, Linha de Base do Escopo, Atualizações dos Documentos do Projeto.

Decomposição
A decomposição é a subdivisão das partes do projeto em componentes menores em que o Gerente do Projeto possa realizar as entregas do trabalho. Estas entregas chamadas de pacotes de trabalho aonde o Gerente de Projeto e equipe de trabalho do Projeto podem estimar o custo e a duração da atividade.
A subdivisão em partes requer uma análise criteriosa do Projeto, existem várias modelos de subdivisões. O PMI® em seu site possui um caderno específico “Practice Standard for Work Breakdown Structures” para o desenvolvimento de uma WBS, as diretrizes para geração, desenvolvimento e aplicação de uma WBS. No caderno podemos obter alguns exemplos e modelos de WBS por setores específicos que necessitam alguns ajustes para o seu projeto.

A profundidade da decomposição depende do tamanho e complexidade do projeto e a necessidade do Gerente de Projetos.


\end{document}
